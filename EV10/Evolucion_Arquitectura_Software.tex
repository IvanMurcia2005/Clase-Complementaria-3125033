
\documentclass[12pt]{article}
\usepackage[utf8]{inputenc}
\usepackage[spanish]{babel}
\usepackage{amsmath}
\usepackage{geometry}
\usepackage{hyperref}

% Configuración de la página
\geometry{a4paper, margin=1in}

\title{Evoluci\'on de la Arquitectura de Software: Microservicios y Transformaci\'on Digital}
\author{
    Carlos Julio Cadena Sarasty \\ 
    Jes\'us Ariel Gonzales Bonilla \\ 
    Ivan Andr\'es Murcia Epia \\ 
    \textit{Servicio Nacional de Aprendizaje SENA, 902ivnmurcia@gmail.com}
}
\date{\today}

\begin{document}
\maketitle

\begin{abstract}
Este art\'iculo explora la evoluci\'on de la arquitectura de software, con un enfoque particular en los microservicios y su papel en la transformaci\'on digital. Se presentan los conceptos clave, herramientas y tecnolog\'ias relacionadas, as\'i como los beneficios y retos asociados con su implementaci\'on. Adem\'as, se destacan estudios de caso en sectores como la educaci\'on, las ciudades inteligentes y el comercio electr\'onico, junto con an\'alisis de su impacto ambiental. Los hallazgos subrayan la importancia de los microservicios como un paradigma arquitect\'onico que impulsa la innovaci\'on, mejora la escalabilidad y fomenta la sostenibilidad.
\end{abstract}

\textbf{Palabras clave:} Microservicios, Arquitectura de Software, Transformaci\'on Digital, Escalabilidad, Sostenibilidad.

\section{Introducci\'on}
En el mundo moderno, la evoluci\'on de la tecnolog\'ia ha transformado la forma en que las organizaciones y los desarrolladores abordan la creaci\'on de software. La arquitectura de software, como disciplina, ha permitido estructurar sistemas que satisfacen las demandas actuales de escalabilidad, eficiencia y sostenibilidad. Dentro de este contexto, los microservicios han surgido como una soluci\'on innovadora, desafiando los paradigmas tradicionales como los sistemas monol\'iticos.

\section{Marco Te\'orico}
\subsection{Introducci\'on a la Arquitectura de Software y Microservicios}
La arquitectura de software desempe\~na un papel crucial en el desarrollo de aplicaciones modernas... (contin\'ua con el contenido del marco te\'orico)

\section{Metodolog\'ia}
Para la construcci\'on de este an\'alisis, se emple\'o una metodolog\'ia basada en la revisi\'on y s\'intesis de literatura especializada... (contin\'ua con el contenido de la metodolog\'ia)

\section{Resultados}
El an\'alisis de las investigaciones revisadas permiti\'o identificar hallazgos clave sobre el impacto y aplicabilidad de los microservicios...

\section{Discusi\'on}
Los hallazgos obtenidos en este an\'alisis evidencian que los microservicios han redefinido la arquitectura de software al ofrecer soluciones innovadoras para los desaf\'ios de escalabilidad y flexibilidad...

\section{Conclusiones}
En conclusi\'on, los microservicios representan una soluci\'on tecnol\'ogica innovadora que se posiciona como una herramienta clave para enfrentar los retos de escalabilidad, flexibilidad y personalizaci\'on...

\section*{Referencias}
\begin{enumerate}
    \item Lewis, J., \& Fowler, M. (2014). Microservices: A Definition of This New Architectural Term. ThoughtWorks. Recuperado de \url{https://martinfowler.com/articles/microservices.html}.
    \item Richardson, C. (2018). Microservices Patterns: With Examples in Java. Manning Publications.
    \item Bass, L., Weber, I., \& Zhu, L. (2015). DevOps: A Software Architect's Perspective. Addison-Wesley Professional.
    \item Newman, S. (2019). Building Microservices: Designing Fine-Grained Systems. O'Reilly Media.
    \item Microsoft Azure. (2021). Designing Distributed Systems with Microservices. Recuperado de \url{https://docs.microsoft.com/en-us/azure/architecture/microservices/}.
\end{enumerate}

\end{document}
